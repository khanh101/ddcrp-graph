\subsection{Node Embedding}

\begin{definition}[Node Embedding]
Given a weighted undirected graph $G = (V, E, w)$ where $V$ is the set of vertices, $E \subseteq V \times V$ is the set of edges, and $w: V \times V \mapsto R^+$ is the edge weight. A node embedding of dimension $d$ is a function $u: V \mapsto R^d$ that maps each vertex to a vector.
\end{definition}

\emph{Deepwalk} \cite{perozzi2014deepwalk} was empirically provable to produce a meaningful representation of nodes for unweighted directed graph. The detailed algorithm is as follow:

\begin{algorithm}[H]
\caption{Deepwalk}
\textbf{Input:}\\
    $G = (V, E, w)$: Undirected weighted graph \\
    $d$: Embedding dimension \\
    $c$: Context size \\
    $\gamma$ Walks per vertex\\
    $l$: Walk length\\

\textbf{Output:}\\
    $X \in R^{|V| \times d}$: Node embedding\\

\begin{algorithmic}
\State \textbf{Initialization:} Sample X
\For{$i$ = $1$ to $\gamma$}
    \State $O$ = Shuffle($V$)
    \For{$v \in O$}
        \State $walk$ = RandomWalk($G$, $v$, $l$)
        \State SkipGram($X$, $walk$, $c$)
    \EndFor
\EndFor
\end{algorithmic}
\end{algorithm}

\begin{algorithm}[H]
\caption{SkipGram}
\textbf{Input:}\\
    $X$: node embedding \\
    $walk$: walk, an array of vertices \\
    $c$: Context size \\

\begin{algorithmic}
\For{$v_i \in walk$} \Comment{$v_i$ is the i-th item in the array}
    \For{$v_j \in walk[i-c, i+c]$} \Comment{array[a, b] is the subarray}
        \State $J(X) = -log(Pr(u'_j | u_i))$
        \State $X = X - \alpha \frac{\partial J}{\partial X}$
        
    \EndFor
\EndFor
\end{algorithmic}
\end{algorithm}

\begin{algorithm}[H]
\caption{RandomWalk}
\textbf{Input:}\\
    $G = (V, E, w)$: Undirected weighted graph \\
    $v_1$: Start vertex \\
    $l$: Walk length\\

\textbf{Output:}\\
    $walk$: walk\\

\begin{algorithmic}
\State $walk = [v_1]$
\While{$walk.length() < l$}
    \State $v = Choose(G, walk[-1])$ \Comment{$array[-1]$ is the last item of the array}
    \State $walk.push\_back(v)$
\EndWhile
\end{algorithmic}
\end{algorithm}

$Choose$ function returns a uniformly random out-going neighbour of the input node. Formally,

\begin{equation}
    ChooseProbability(v_j | v_i) \propto \left\{
        \begin{array}{ll}
            1 \;\;\;\; \text{if $(v_i, v_j) \in E$}\\
            0 \;\;\;\; \text{otherwise}
        \end{array}
    \right.
\end{equation}

Probability of context $j$ given vertex $i$ is given by \cite{mikolov2013distributed}:

\begin{equation}
    Pr(u'_j|u_i) = \frac{exp(u_j^{'T} u_i)}{\sum_{k=1}^{|V|} exp(u_k^{'T} u_i)}
\end{equation}

Where $u'_i$ and $u_i$ are the embedding of vertex $v_i$ when it is treated as "context" and "vertex" respectively.


\newpage
\subsection{Community Detection}
\begin{definition}[Community Detection]
Given a graph $G = (V, E)$ and a node embedding $X \in R^{n \times d}$. Define a partition $P$ of $V$ is a set of disjoint non-empty subsets of $V$ such that union of $P$ elements is $V$.

$P = \{P_k\}_{k=1}^{K}$ where
\begin{itemize}
\item $P_k \neq \emptyset$
\item $\cup_{k=1}^{K} P_k = V$
\item $P_i \cap P_j = \emptyset$ for all $P_i \neq P_j$
\end{itemize}

A community detection algorithm returns a partition of vertices.
\end{definition}

Different from the traditional \emph{Chinese Restaurant Process} (\emph{CRP}), in the work of  David M. Blei and Peter I. Frazier on \emph{distance dependent Chinese Restaurant Process}, \emph{ddCRP} \cite{blei2011distance} generalized the sitting arrangement of customers by a unweighted directed graph. Each vertex is corresponding to a customer, each weakly connected component is corresponding to a table. In this model, each vertex either links to another vertex or links to itself. Let $z$ be the vector represent the linking process such that $z[i]$ is the customer that customer $i$ links to. Then, \emph{ddCRP} defines the conditional probability as follow \cite{blei2011distance}:

\begin{equation}
     P(z[i] = j \;|\; z[0:i-1], \alpha) \propto \left\{
    \begin{array}{ll}
        \alpha \;\;\;\;\;\;\;\;\;\;\; \text{if $j = i$}\\
        f(i, j) \;\;\;\; \text{if $j \neq i$}\\
    \end{array}
    \right.
    \label{eq:ddcrp}
\end{equation}

Where $\alpha$ is the concentration parameter and $f(i, j)$ is defined as the decay function. The decay function controls the probability of two customers sitting in the same table. If the decay function equals 1 for every pair of customers, the model is equivalent to the traditional \emph{CRP}.

Let $x$ be the observations. In Gibbs sampling, each latent variable is sampled given all other latent variables being observed. The Gibbs sampling probability is as thus \cite{blei2011distance}:

\begin{equation}
    P(z[i] = j | (z - z[i]), \alpha, x, G_0) \propto P(x | (z - z[i] + j), G_0) P(z[i] = j | (z - z[i]), \alpha)
    \label{eq:gibbs_fac}
\end{equation}

Where $(z - z[i])$ denotes the customer assignment without $z[i]$,  $(z - z[i] + j)$ denotes the customer assignment where $z[i]$ is replaced by $j$. $G_0$ denotes the prior distribution on $x$.

The first term is the likelihood of data given the new customer assignment $(z - z[i] + j)$ and prior $G_0$. The second term  is \emph{ddCRP} process probability (equation \ref{eq:ddcrp}). The authors of \cite{blei2011distance} factorized the first term into a product of each table likelihood.

\begin{equation}
    P(x | (z - z[i] + j), G_0) = \prod_{t \in T} P(x_t | G_0)
\end{equation}

Where $T$ is the set of tables. This leads to an efficient Gibbs sampling scheme:

\begin{equation}
     P(z[i] = j \;|\; (z - z[i]), \alpha, x, G_0) \propto \left\{
    \begin{array}{ll}
        \alpha \;\;\;\;\;\;\;\;\;\;\;\;\;\;\;\;\;\;\;\;\;\;\;\;\;\;\;\;\;\;\;\;\;\;\;\; \text{if $j = i$}\\
        f(i, j) \;\;\;\;\;\;\;\;\;\;\;\;\;\;\;\;\;\;\;\;\;\;\;\;\;\;\;\;\; \text{if $j \neq i$ and no table join}\\
        f(i, j) \frac{P(x_{t_{ij}} | G_0)}{P(x_{t_i} | G_0) P(x_{t_j} | G)} \;\;\; \text{if $j \neq i$ and  table join}\\
    \end{array}
    \right.
    \label{eq:gibbs}
\end{equation}

Where $t_i$ and $t_j$ is the table of customer $i$ and $j$, $t_{ij}$ is the join table.

\newpage
\subsection{Cluster Ensemble}

\begin{definition}[Meta-Graph \cite{strehl2002cluster}]
Given a set of $U$ of $n$ elements, a clustering result is defined as a family of clusters/subsets of $U$: $C \subseteq \mathcal{P}(U)$. Each meta-node is defined as an element $c$ of $C$. Construct meta-edge by the Jaccard similarity (intersection over union): $w_{i, j} = \frac{|c_i \cap c_j|}{|c_i \cup c_j|}$
\label{def:meta_graph}
\end{definition}


 Alexander Strehl and Joydeep Ghosh \cite{strehl2002cluster} introduced an algorithm to ensemble clustering results namely MCLA algorithm based on the concept of the \emph{Meta-Graph}. The algorithm consists of four major steps.
 \begin{enumerate}
     \item Construct the \emph{Meta-Graph} from clustering.
     \item Partition the \emph{Meta-Graph} into K-balance meta-clusters.
     \item Construct the probability of a given node belongs to a meta-cluster by averaging all clusters.
     \item For each node, decide its corresponding meta-cluster based on the calculated probability.
 \end{enumerate}
 
 
 \newpage
 \subsection{Data marginal likelihood}
 
Assuming data points distribute according to a Gaussian, choosing the Normal-inverse-Wishart (NIW) prior, the data marginal likelihood is hence calculated according to Kevin P. Murphy work \cite{murphy2007conjugate} (page 21).

\begin{equation}
    P(\mu, \Sigma | D) = NIW(\mu, \Sigma | m_N, k_N, v_N, S_N)
\end{equation}
\begin{equation}
    k_N = k_0 + N
\end{equation}
\begin{equation}
    v_N = v_0 + N
\end{equation}
\begin{equation}
    m_N = \frac{k_0 m_0 + N\overline{x}}{k_N}
\end{equation}
\begin{equation}
    S_N = S_0 + S + k_0 m_0 m_0^T - k_N m_N m_N^T
\end{equation}

\begin{equation}
    P(D) = \frac{1}{\pi^{Nd/2}} \frac{\Gamma_d(v_N/2)}{\Gamma_d(v_0/2)} \frac{|S_0|^{v_0/2}}{|S_N|^{v_N/2}} (\frac{k_0}{k_N})^{d/2}
    \label{eq:table_likelihood}
\end{equation}

Where $S\overset{\Delta}{=} \sum_{i=1}^N x_ix_i^T$ as the uncentered sum-of-squares matrix.